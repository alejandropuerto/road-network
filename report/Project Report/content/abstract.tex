\chapter{Abstract}
\label{cha:abstract}

Mérida urban street network has not been studied in a network theory-based background. This study aims to fill that gap by obtaining and characterizing the urban street network of Mérida at a municipal and basic urban geostatistical area (AGEB) scales, as well as identifying AGEBs with similar characteristics. It presents findings on Mérida urban form and street network characteristics, emphasizing relevant measures of network theory and morphology such as structure, density, centrality, and resilience. The results showed a prevalence of 3-way intersections in Mérida, including its AGEBs, and that planning eras and design paradigms of urban street road influence the density, resilience and total street length of the network. Finally, we found a good definition of the 14 subdivisions of the city created by an agglomerative clustering algorithm. All this work extents for reproducibility as all the data, environment, street networks, and measures have been shared in a repository for other researchers to use.