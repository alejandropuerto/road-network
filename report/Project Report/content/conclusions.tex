\chapter{Conclusions and recommendations}
\label{cha:conclusions}
\section{Conclusions and recommendations}

In this work, we analyze and characterize the municipality of Mérida, Yucatán, México at a municipal and AGEB scale to have a general and detailed view of the street network design. We believe it is the first local work of its kind which allow us acquire better understanding of the topological structure of street networks. We find that planning eras and design paradigms of urban street roads on Mérida influence street network density, resilience, and total street length. Overall, Mérida is characterized by the importance of 3-way intersections. 

We find a significant linear relationship between total street length and the number of nodes in a network. Also, most AGEB networks typically follow right-skewed distributions of street segment length.

Through centrality measures we identify important nodes in different use cases, and how they can be seen as means to analyze resilience and robustness of the network. 

Clustering findings reveal the spatial correlation of AGEBs within the city and how not only topological attributes but also socio-demographic data influences the clear subdivision of the city in different parts (14 clusters). The subdivisions created by the agglomerative clustering algorithm seems to have a natural definition of the parts from the city. However, we recommend finding the ideal number of clusters based on efficient and official techniques to observe the difference.


\section{Future work}

This work was only focused on Mérida street network, but we also have an ongoing work for all urban streets of all cities in México. This will lead to a more deep understanding and wider range of interest.

We hope that in this and further work, we can obtain more useful insight and gain more understanding on street network analysis; as well as helping government and private industry to consider take the analysis into account in their urban development planning.